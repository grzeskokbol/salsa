\chapter{System installation}
In this chapter we describe how to set up a SALSA system from scratch.
Currently only the software part is included, the hardware part will be added
later.

\section{Software}
This log was written by Eskil Varenius 2015-11-16 as he re-installed all
software on the computer “vale” from scratch. Below follows the polished log,
where I have tried to improve the clarity of the text for easy reading. All
commands in the terminal are in bold face. In general things should work as
described, but there may be one or several reboots of the computer missing to
read in config files etc. If you get stuck, try a reboot. This manual may also
be useful for a “client” computer like “brage” which is not running the booking
system server. In that case, large parts can be omitted, in particular setting
up mysql and apache servers, and only a mysql client should be needed.

\subsection{Pre-requisites}
At the start of this, I had one completely clean computer (Dell Optiplex 760)
and one Live-DVD of Linux Mint 17.2 Mate 64-bit. I also had a backup of the
directories “/opt” and “/var/www” as well as a SQL-dump of the drupal database.

\subsection{Installing Linux Mint 17}
First I plugged in the internet cable and inserted the live-DVD. I had to press
F12 when starting the system to get the boot menu, then wait a minute until I
was sure that the computer had accepted the DVD, and then press “Boot from dvd”
option. After some time, the Linux Mint Live environment was up and running,
and I used the “Install Linux Mint” option on the desktop. I ran the guide with
default options, since the hard-drive was clean no re-formatting etc. was
needed. When the install finished the screen was black, probably because I left
it long enough for the screen-saver timer to start, but no screen save was
active. I had to press space on the keyboard to wake it up, and it then asked
me if I wanted to restart or continue with Live DVD. I choose restart and was
then told to eject install media and close tray. I choose \verb!salsa_admin! as user
and choose a new password to reflect the year :). The system now booted up all
fine. First thing I did was to choose not to display the startup dialogue all
the time. Then I ran a complete update using the graphical software updater.
Then I rebooted the system. To remove Mint Welcome screen for new users, which
is only confusing to describe in the instructions for remote login, I ran sudo
apt-get purge mintwelcome. Finally I installed the editor “vim” which I like to
use to edit config files etc. by typing sudo aptitude install vim. 

\subsection{Network configuration}
To make Vale resolve other local computers, as being able to write “ssh brage”
instead of the “ssh brage.oso.chalmers.se”, manual config was needed. The file
\verb!/etc/resolvconf/resolv.conf.d/base! was edited to contain the following four
lines
\begin{verbatim}
domain oso.chalmers.se
search oso.chalmers.se
nameserver 129.16.1.53
nameserver 129.16.2.53
\end{verbatim}

\subsubsection{Setting manual IP}
The SALSA computers have two network cards: one to the outside world (public),
and one used for the USRP/RIO units (local). The only important requirement is
that for the USRP (local) the card has to do 1Gbps (i.e. 100Mbps is not
enough). For the public network, vale has a dedicated IP-address, so I changed
to manual IP using the graphical tools: “Control center-> network connections
-> Wired connection 1 -> IPv4”. For easy reference I also changed the name of
this connection to “Public internet”, as in connected to the outside world. The
IP was set to “Manual” and one adress line was added (IP, netmask, gateway) as
129.16.208.184, 255.255.255.0, 129.16.208.1. (Brage has 129.16.208.20). I did
not specify any DNS. For the local card I needed two lines of adresses since
the USRP and RIO boxes were different: 192.168.10.10, 255.255.255.0,
192.168.10.1 for USRP, and 169.254.212.10, 255.255.255.0, 169.254.212.1 for
RIO.  

\subsection{Setting up SSH-access}
To be able to login remotely to Vale via SSH, I installed “open-ssh server”:
sudo aptitude install open-ssh server and no further configuration was needed. 
\subsection{Install webserver and mysql database}
I used apache2 as web server. Install by sudo aptitude install apache2. Then I
replaced the contents of  /var/www/html (only default “index.html”) with the
backed up files, i.e. index.html and “salsa”-folder. I made sure of permission
by running sudo chown -R www-data:www-data salsa/ in the /var/www/html-folder.
Then I installed a mysql server by sudo aptitude install mysql-server and when
prompted I choose the root password as the standard admin password. Then
I installed phpmyadmin to handle mysql administration sudo aptitude install
phpmyadmin, where I selected apache2 using space key and set the password to
the admin password. I also installed mysql client, not sure if it was
needed, as sudo aptitude install mysql-client. Finally, phpmyadmin complained
that so I solved it by running sudo php5enmod mcrypt and sudo service apache2
restart.  
\subsection{Creating MySQL users and tables}
Now I logged in to vale.oso.chalmers.se/phpmyadmin as root. First I added the
user \verb!salsa_drupal!, with hostname “localhost” and selected the option “Create
database with same name and grant all privileges”. Note that the username and
password chosen here has to be match in the file
“/var/www/html/salsa/sites/default/settings.php” line 567 for drupal to work,
and in the crontab bash script for the handling of bookings to work. I now want
to import the old database structure, but before doing so in phpmyadmin we need
to allow large files to be imported, since the SQL-dump is around 240Mb. To up
the limits, we edit the file “/etc/php5/apache2/php.ini” to say
\begin{verbatim}
upload_max_filesize = 512M
memory_limit = 512M
post_max_size = 512M
\end{verbatim}
, and then we restart apache as sudo service apache2 restart. Now the import
worked fine and the database was imported, along with the \verb!salsa_archive!
table. I now created the \verb!salsa_archive! user and opened privileges for this user
only on the archive table, using phpmyadmin “edit privileges”. This was
done only for localhost access, and nicely returned the message 
\verb! “GRANT SELECT , INSERT , UPDATE , REFERENCES ON! \newline
\verb!`salsa_drupal`.`salsa_archive` TO 'salsa_archive'@'localhost';! 
as confirmation.  
\subsubsection{Enable access to booking system}
To be able to read the mysql-database from
another computer (i.e. not localhost) I need to add another user with name
\verb!salsa_drupal! but specify “host=brage.oso.chalmers.se” for brage. But, for
this security layer to be considered, I first have to allow access from
anything else than the default 127.0.0.1. this is done by editing the file
“/etc/mysql/my.cnf” (on the server computer vale) and commenting the line
“bind-adress=127.0.0.1” followed by sudo service mysql restart. NOTE: this was
also needed for the \verb!salsa_archive! user to be able to send the new fits files.

\subsection{Setting up Apache and Drupal}
For drupal to work we need to make sure the login credentials are OK in the
file “settings.php”. I also made sure the settings.php file had proper
permissions as sudo chown go-rwx settings.php. Just checking the webpage showed
that I could not access anything except for the front page. Several issues had
to be sorted out. It seems Drupal needs the “gd” library, so we need to install
it by sudo aptitude install php5-gd and then sudo service apache2 restart. I
then found out that drupal needs \verb!mod_rewrite! enabled in apache2, so I did
sudo a2enmod rewrite and sudo service apache2 restart to enable it. After this
I needed to modify “/etc/apache2/apache2.conf” to allow overrides, else the
CleanURL-config in Drupals .htaccess file will not work. So I changed the
section on /var/www/ in the apache config file to say “AllowOverride All”
instead of “none”, i.e.  
\begin{verbatim}
<Directory /var/www/>
       Options Indexes FollowSymLinks
       AllowOverride All
       Require all granted
</Directory>
\end{verbatim}
For future drupal maintenance, the software “drush” is very useful, so I
installed it as sudo aptitude install drush. Finally I changed the password of
the web-admin account.  

\subsection{Install and configure Cendio ThinLinc}
This is a wonderful remote desktop service which is free for less than 10
simultaneous users, perfect with SALSA. First I requested the latest version on
the cendio webpage and got an email where I could download the file
“tl-4.5.0-server.zip”. This was unzipped and I ran the graphical installer
“Thinlinc Server installer”. Following instructions, this also runs a setup
utility afterwards, where I choose password for the web admin tools etc. The
utility installs a lot of missing dependencies, the only thing I skip is to
configure printers.  ThinLinc can restrict user access in terms of groups. It
is easy to modify the group membership of users via the cronjob script, so I
created a group where I place the current user(s) who can login to the computer
using the ThinLinc webaccess as \verb!sudo addgroup salsa_weblogin!. This is
then read by ThinLinc after modifying the file
\begin{verbatim}
/opt/thinlinc/etc/conf.d/vsmserver.hconf
\end{verbatim}
where I set
\verb!allowed_groups=salsa_weblogin!.  Now I want to tailor the software to
SALSA. First is to remove the welcome windows by setting
\verb!show_intro=false! in 
\begin{verbatim}
/opt/thinlinc/etc/conf.d/profiles.hconf
\end{verbatim}
I also replaced “Cendio thinlinc” with “SALSA Vale” in short welcome lines.
Then I want to make the login-page that users see a bit more SALSA-like. This
page is reached at “vale.oso.chalmers.se:300”. First I edit the welcome page by
editing the file
\begin{verbatim}
/opt/thinlinc/share/tlwebaccess/templates/main.tmpl
\end{verbatim}
with custom replaced html. It should be possible to use the templates in a
better way, rather than replacing , but this works. I also also added logo.png
to the folder
\begin{verbatim}
/opt/thinlinc/share/tlwebaccess/www/images
\end{verbatim}
to show SALSA logo on the login page. Now ThinLinc was ready for use!

\subsection{Setting up the cronjob}
The SALSA booking system works by crontab running a bash-script every minute to
check for new bookings. This script needs some software, e.g. “members” which
was installed as sudo aptitude install members. From the backup directory (or
from github repo with appropriate modifications) I then copied the crontab
files to /opt/salsa.  I also copied the “skel”-files which will be used by the
cron-script to place desktop shortcuts to the control program. For this, an
icon was also included in the /opt/salsa directory. I also needed to add the
group \verb!salsa_users! by running \verb!sudo addgroup salsa_users!.  The crontab could
now be ran, and was started by using sudo crontab -e, and adding the line:
\begin{verbatim}
*/1 * * * * /opt/salsa/crontab/update_salsa_access_from_drupal.sh
\end{verbatim}
at the bottom of the file.



\subsection{Installing the SALSA control software}
The SALSA control software can be obtained from Eskil Varenius’ github:
\begin{verbatim}
git clone https://github.com/varenius/salsa.git salsa.git
\end{verbatim}
The files in the folder \verb!/Control_program! was now copied to
/opt/salsa/controller. The default config file was renamed to SALSA.config and
then edited with computer/telescope specific settings. To make SALSA easily
accessible by all users, the global configuration file “/etc/profile” was
edited to append the PATH command 
\begin{verbatim}
export PATH=$PATH:/opt/salsa/controller/
\end{verbatim}

To make the SALSA python software run, the following was needed
sudo aptitude install python-tk python-scipy python-astropy python-mysqldb python-dev python-matplotlib python-pip
Then, using pip (installed above) the following two python modules were installed
sudo pip install pyephem;
sudo pip install tendo;


Make sure that the SALSA item in the /opt/salsa/controller is a symbolic link.
Sometimes, when copying, it may be replaced with an actual python file. This
will work, but then when updating the software it will not be updated and
problems may arise. To create the symbolic link to main.py, use sudo ln -s
main.py SALSA.

The SALSA program can now be started using SALSA in a terminal (after logging
out and in again), or via /opt/salsa/controller/main.py.

Done!

\subsection{Backup}
We want SALSA to be backed up. Currently this is done via root-access using a
public ssh key which I got from Glenn Persson in Onsala. By default, root-login
was enabled “without-password” in ssh-config so no changes needed apart from
adding the public key to \verb!/root/.ssh/authorized_keys!. Note that the folder .ssh
and files had to be created. Permissions were set using \verb!chmod 700 .ssh! and
\verb!chmod 600 .ssh/authorized_keys!. Backed up folders (as of 2015-11-17) are
\begin{verbatim}
/home/salsa_admin/ 
/opt/salsa/ 
/opt/thinlinc/ 
/var/lib/mysql/
/var/www/
\end{verbatim}

\subsection{Setting up GalilTools for RIO development}
To develop new features it is useful to communicate directly with the RIO using
the software GalilTools available from Galil. Downloaded from
\begin{verbatim}
http://www.galil.com/download/software/galiltools/linux/
\end{verbatim} 
I used the file
\verb!galiltools_1.6.4_amd64.deb! from the folder Ubundu 14.04.  and installed
with \verb!sudo dpkg -i galiltools_1.6.4_amd64.deb! and then i could start the
software using only “galiltools” in a terminal.

\subsection{Qt-designer for graphical UI development}
I use the graphical designer software called qt4-designer to construct the
SALSA GUI. It is installed by typing sudo aptitude install qt4-designer.
Then, the SALSA user interface is edited by typing \verb!designer SALSA_UI.ui!.
Once saved, the designer file needs to be converted to python code. This
is done using the pyuic tool, which first has to be installed through
\verb! sudo aptitude install pyqt4-dev-tools!. Then, one may run
\verb! pyuic4 SALSA_UI.ui > UI.py! to update the UI.py file. Possible hooks
for buttons etc. are implemented manually in main.py.

\section{Obtain HTTPS certificates}
This section describeds How to install HTTPS certificates for SALSA. This
is needed to avoid warnings when connecting through the ThinLinc HTML5 
browser client. The following commands are a mix of brage/vale commands.
Make sure to change the compute name accordingly, e.g. to either brage
or vale in all such references.

\subsection{Create and install certificates}
We follow the instructions at https://letsencrypt.org/, i.e. us the certbot 
script.
\begin{verbatim}
sudo mkdir /opt/certbot
cd /opt/certbot
sudo wget https://dl.eff.org/certbot-auto
sudo chmod a+x certbot-auto
sudo ./certbot-auto
\end{verbatim}
Answer yes to install packages. Add email salsa.onsala@gmail.com. Choose not to
share email adress. Agree to terms of service.  Enter domain
“brage.oso.chalmers.se” or “vale.oso.chalmers.se” Finally, when prompted,
choose alternative 1: Easy - Allow both HTTP and HTTPS access to these sites.

\subsubsection{Set correct file permissions}
For some reason, certificate file permissions are set to 644
instead of 600 by the certbot script. ThinLinc requires 600 to run, so 
we change:
\begin{verbatim}
sudo chmod -R 600 /etc/letsencrypt/archive/brage.oso.chalmers.se/
\end{verbatim}

\subsection{configure ThinLinc}

\subsubsection{Edit Thinlinc config files}
To use cert files, edit
\verb!/opt/thinlinc/etc/conf.d/webaccess.hconf!
to say 
\begin{verbatim}
cert=/etc/letsencrypt/live/vale.oso.chalmers.se/fullchain.pem
certkey=/etc/letsencrypt/live/vale.oso.chalmers.se/privkey.pem
\end{verbatim}

To make sure hostname is correct (else thin linc will redirect
to use IP adress, for which cert is not valid),
edit the file \verb!/opt/thinlinc/etc/conf.d/vsmagent.hconf!
to say 
\begin{verbatim}
master_hostname=brage.oso.chalmers.se
agent_hostname=brage.oso.chalmers.se
\end{verbatim}

\subsubsection{Restart ThinLinc}
ThinLinc needs to be restarted to read new configuration files.
\begin{verbatim}
sudo service vsmserver restart
sudo service vsmagent restart
sudo service tlwebadm restart
sudo service tlwebaccess restart
\end{verbatim}
NOTE: If the above do not solve the issue, try rebooting to make sure all is
restarted.

\subsection{Restart Apache}
Apache needs to be restarted to red new config:
\begin{verbatim}
sudo service apache2 restart
\end{verbatim}

\subsection{Add crontab entry to update certificates}
Certificates from Lets Encrypt are valid for 90 days. To automatically
update the certificates, we add a cronjob by running
\verb!sudo crontab -e! and adding
\begin{verbatim}
# Update httos-certificates at 02:23 every night
23 2 * * * /opt/certbot/certbot-auto renew --quiet --no-self-upgrade
24 2 * * * chmod -R 600 /etc/letsencrypt/archive/vale.oso.chalmers.se/
\end{verbatim}
Second line needed because else thinlinc client won’t start, as cert-files have read permissions.

\subsection{Troubleshooting}
If you need to remove all Let's encrypt files, e.g. to start over,
run 
\begin{verbatim}
rm -rf /etc/letsencrypt /var/log/letsencrypt 
/var/lib/letsencrypt ${XDG_DATA_HOME:-~/.local/share}/letsencrypt
\end{verbatim}

\section{Hardware}
\subsection{RIO}
No drivers or files are needed for the RIO because we are using telnet for
that. We can however install the \emph{GalilTools} software to communicate
with the RIO for e.g. troubleshooting.

\subsection{USRP}
For the USRP to work we first need drivers. These can be installed by
To get GNUradio and USRP things: 
\begin{verbatim}
sudo aptitude install gnuradio libuhd-dev libuhd003 uhd-host
\end{verbatim}

The USRP also need some extra settings, else it will complain and suggest one
to run the commands 
\verb!sysctl -w net.core.rmem_max=50000000! and 
\verb!sysctl -w net.core.wmem_max=1048576!. But, this only works
temporarily, I need a permanent fix for all users. From the USRP documentation
it is said that this could be fixed by editing the file
“/etc/sysctl.d/uhd-usrp2.conf” to contain
\begin{verbatim}
# USRP2 gigabit ethernet transport tuning
net.core.rmem_max=50000000
net.core.wmem_max=1048576
\end{verbatim}
While doing this, we can also get rid of the USRP warning “Unable to set the
thread priority.” by adding, at the end of the file
“/etc/security/limits.conf”, the following lines: 
\begin{verbatim}
@salsa_admin    -    rtprio    99
@salsa_users    -    rtprio    99
\end{verbatim}

\subsubsection{USRP N200 firmware update}
If the USrp comlains that FIRMWARE FPGA is too old, e.g. with \emph{Please update the firmware and FPGA images for your device}, then update the firmware. This can be done in two steps. First, run
\begin{verbatim}
/usr/lib/uhd/utils/uhd_images_downloader.py
\end{verbatim}
and then run 
\begin{verbatim}
uhd_image_loader --args="type=usrp2,addr=192.168.10.2,reset"
\end{verbatim}
where the IP is the IP of the USRP which needs to be updated. Instructions
taken from \verb!https://files.ettus.com/manual/page_usrp2.html!
