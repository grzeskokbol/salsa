\chapter*{Abstract}
\addcontentsline{toc}{chapter}{Abstract}
SALSA-Onsala (``Such A Lovely Small Antenna'') is a 2.3~m diameter radio
telescope built at Onsala Space Observatory, Sweden, to introduce pupils,
students and teachers to the marvels of radio astronomy.  The sensitive
receiver makes it possible to detect radio emission from atomic hydrogen far
away in our galaxy. From these measurements we can learn about the kinematics
and distribution of gas in our galaxy, the Milky Way. One can also use
the antenna for other projects which does not involve hydrogen. In this
document we describe how you can measure the antenna response function,
also called the \emph{beam}, of the SALSA telescope by observing the 
total power received from the Sun.

First we review some basic concepts of how radio telescopes work and
what the antenna reponse function for SALSA is expected to look like. 
Then we describe how to use the SALSA control program to observe the
Sun to learn about the beam of SALSA. 

Please note that this document is focused on understanding the antenna
response and only briefly describes the telescope control program. 
Instructions for operating the SALSA telescope can be found in the document
entitled \emph{SALSA users manual} available at the SALSA website.


\vspace{9cm}




{\bf Cover image:} The SALSA telescopes in Onsala.
