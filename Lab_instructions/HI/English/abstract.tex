\chapter*{Abstract}
\addcontentsline{toc}{chapter}{Abstract}
SALSA-Onsala (``Such A Lovely Small Antenna'') is a 2.3~m diameter radio
telescope built at Onsala Space Observatory, Sweden, to introduce pupils,
students and teachers to the marvels of radio astronomy.  The sensitive
receiver makes it possible to detect radio emission from atomic hydrogen far
away in our galaxy. From these measurements we can learn about the kinematics
and distribution of gas in our galaxy, the Milky Way. 

In this document we first review some properties of the Milky Way, starting by
describing the Galactic coordinate system and the geometry of a rotating disk.
Then, we describe how to use data from SALSA to understand how fast gas rotates
at different galactic radii, i.e. how to make a \emph{rotation curve}. Finally,
we use additional measurements, and our knowledge of the kinematics, to make a
map of the spiral arms.

Please note that this document is focused on the scientific interpretation. 
Instrutions for operating the SALSA telescope can be found in the document
entitled \emph{SALSA users manual} available at the SALSA website.


\vspace{9cm}




{\bf Coverimage:} Artist's conception of the spiral structure of the
Milky Way with two major stellar arms and a central bar. Distances in
light-years (ly) and directions in galactic coordinates. Credit:
NASA/JPL-Caltech/ESO/R. Hurt.
