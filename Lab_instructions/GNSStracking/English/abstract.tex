\chapter*{Abstract}
\addcontentsline{toc}{chapter}{Abstract}
SALSA ('Such A Lovely Small Antenna') are two 2.3-m-diameter radio
telescopes built at Onsala Space Observatory in Sweden to introduce pupils,
students and teachers to the marvels of radio astronomy. This receiving system allows users to detect 
radio emission from atomic hydrogen located far away in our galaxy, the Milky Way. 
SALSA can also be utilized in other types of observations. In the following document, we describe how SALSA can 
be used to investigate spectra of signals emitted by satellites that are part of the Global Navigation Satellite Systems (GNSSs).\par{}

First, we give a brief introduction to GNSS as well as highlight its common applications. 
Next, we describe the structure of GNSS signals and explain the basic principles of the global positioning through the use of GNSS. 
Finally, we give a detailed account in words on how to use SALSA to detect signals transmitted by GNSS satellites. \par{}
Please note that this document focuses only on one application of SALSA. General instructions on how to operate telescopes
can be found in the document entitled \emph{SALSA users manual} available at the 
\href{https://vale.oso.chalmers.se/salsa/support}{SALSA} website.


\vspace{9cm}




{\bf Cover image:} A GALILEO satellite (upper, credit: European Space Agency) and SALSA telescopes (lower, credit: Grzegorz Klopotek).
