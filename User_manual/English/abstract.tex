\chapter*{Abstract}
\addcontentsline{toc}{chapter}{Abstract}
SALSA-Onsala (``Such A Lovely Small Antenna'') is a 2.3~m diameter radio
telescope built at Onsala Space Observatory, Sweden, to introduce pupils,
students and teachers to the marvels of radio astronomy.  The sensitive
receiver makes it possible to detect radio emission from atomic hydrogen far
away in our galaxy. From these measurements we can learn about the kinematics
and distribution of gas in our galaxy, the Milky Way.

In this document we describe in detail how to operate the telescope
and how to extract information from the data files you obtain with SALSA.
We also include a summary of the technical capabilites and limitations
of a SALSA telescope.

Please note that this document does not include any scientific interpretation.
A guide to interpreting your measurements can be found in the documents 
describing the possible projects on the SALSA website, for example the project
\emph{Mapping the Milky Way}.

\vspace{9cm}




{\bf Coverimage:} Image of the SALSA telescopes in Onsala.
